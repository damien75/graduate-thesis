\chapter{Introduction}

The technological improvement of noninvasive medical imaging techniques in the past couple of decades has allowed us to get a better understanding of the heart's structure \cite{sinusas2008multimodality}. This is also aided by the increasing amount of imaging data that can be collected more and more easily and with better definition as time goes by.

As the imaging techniques get better \cite{shaw2010cardiovascular}, their analysis begins to play a major role in our understanding of the causes and effects of heart disease. Cardiovascular diseases are still the leading global cause of death \cite{mozaffarian2015heart} since 1900 (except for the year 1918) and the knowledge gained by cardiac image analysis has to be used in a clever way to drive them down.

To this day, the most common imaging modalities used by clinicians to get images of the heart are Magnetic Resonance Imaging (MRI), Computed Tomography (CT), and echocardiography. Thus far little work has been carried out in reconstructing models of heart wall fibers and their geometry. Our imaging tool tries to come up with a solution to this heart wall fiber approach. This latter problem is important since it determines heart wall mechanical function \cite{hooks2002cardiac}.

The core organization of these fibers can be altered by various common pathologies, including rheumatic heart disease or inflammatory heart disease or dilated, hypertrophic, restrictive cardiomyopathies, all of which can lead to a deficient contraction of heart wall muscle \cite{kerwin2000ventricular, beg2004computational}.

Numerous former histological studies have shown how fibers in a healthy heart are smoothly wrapped around it and that the variation of the helix angle (a measure of fiber orientation) is a common descriptor of local fiber geometry \cite{geerts2002characterization}. Computationally aided visualization of the disorganization that can occur in the heart structure could be helpful for modern treatments that feature tissue engineering methods to restore the contractile properties of the heart \cite{caplan2006mesenchymal,laflamme2007cardiomyocytes,laflamme2005regenerating, song2012heart, zimmermann2004engineered}, or in more extreme cases where they proceed with reconstruction, or even ablation of infarcted regions \cite{athanasuleas2004surgical, di2001effects, jones2009coronary, sartipy2005dor}.

In general the most common means to assessing the impact of a heart disease is via echocardiography, computed tomography (CT) and MRI. CT is a good tool to analyze patients with coronary artery diseases as this tool is efficient in revealing different layers in a slice of a body and therefore highlighting foreign objects which for instance could obstruct an artery. Echocardiography, the most often used method to analyze heart disease, can provide numerous useful sources of information about heart shape and size, the flow of input and output blood, but as presently used in the clinic it does not provide information on tissue organization and fiber structure. Although research groups focus on getting fiber orientation from ultrasound, it is still a work in progress and not yet used in clinical practices.

The imaging of patient specific models of heart wall fibers has become popular using Diffusion MRI methods and has allowed for the investigation of cardiac myofibers and laminar architecture \cite{rademakers1994relation, helm2005measuring, rohmer2007reconstruction}. Despite this enthusiasm, most work so far has focused on healthy hearts. Although some work has been done on fiber modeling \cite{pami2015, savadjiev2012heart} there are very few methods that provide subject specific differential geometric signatures of fiber orientation.

\section{Objectives and Organization of this Thesis}

The objective of this thesis is to demonstrate the potential of Maurer-Cartan connection forms to both characterize healthy tissue and indicate regions affected by infarcts via an analysis of their error of fit. This analysis can reveal regions of healthy tissue but also can indicate partially viable tissue in the infarct zone, which is an information that is not directly revealed by ADC or FA scalar measures.

In Chapter 2 we provide a literature review where we will give an overview of the imaging methods used and how relevant they can be in our case, and what kind of information we can obtain from these technologies. Then we will review current knowledge of heart fiber geometry and demonstrate that Maurer-Cartan connection forms show promise for describing local fiber orientations. Then we will try to understand how infarction can scar a region of the heart, which will be the first source of major modification in the heart wall organization, and then remodels itself to lead to a unique geometrical organization of the fibers in the infarct region. Finally the experimental setup will give an overview of the data we were able to test on and how it was obtained, as well as histopathology in some cases that can help us testify for the actual heart wall fiber organization.

In Chapter 3, we start from our knowledge of Maurer-Cartan connection forms and use this model to get the most accurate description of the fibers of the heart \cite{pami2015} that we can get. Then we explain how our fitting method works to get those parameters from the data and how our error of fit measurements can indicate healthy regions with a coherent fiber orientation throughout the heart wall or confirm the presence of scar tissue and collagen that can also be observed on the ADC or FA maps.

In Chapter 4, our results are explained and shown with helpful illustrations. There we try and analyze the extra information that our fitting method and error of fit measurements were able to give us. We compare the results we obtain with our approach to the ADC and FA scalar maps that are usually referred to, in order to be sure that we get at least similar information about the infarct. Furthermore, the error of fit measurement can show more precisely areas with a high deficiency of coherence in fiber orientation and patches of fibers that still seem to be properly organized to help with the contractile function of the heart.

In the Conclusion, we will see that our approach shows promise and more work can be done in this direction to collect more quantitative results. This would allow this method to be more appealing if it indeed comes with useful information that is otherwise hard to obtain in a non invasive approach.