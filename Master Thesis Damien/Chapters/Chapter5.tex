\newcommand{\degree}{^{\circ}}

\chapter{Results and Discussion}

We used an established Rician smoothing method to deal with noise in the diffusion images \cite{wiest2008rician}. This is a type of non-local smoothing technique which uses voxel to voxel similarity to guide the smoothing process. In general the parameters for this filtering method must be tuned to prevent oversmoothing, which can happen for instance if the weight is not sufficiently penalized when the similarity is not high enough. We used the publicly available MedInria software to carry out the filtering, and then to reconstruct the diffusion tensor from the raw diffusion weighted scans, from which fiber orientations were extracted as the first principal eigen vector. In practice we used a threshold on the FA map as a mask to restrict further processing.

\section{Quantitative Results}

\begin{figure}
    \centering
    \begin{subfigure}{1\textwidth}
        \centering
        \includegraphics[width=\textwidth]{figures/histogram_pig6_no_smooth}
        \caption{Without any smoothing}
        \label{fig:histogram_pig6_no_smooth}
    \end{subfigure}
    \begin{subfigure}{1\textwidth}
        \centering
        \includegraphics[width=\textwidth]{figures/histogram_pig6_smooth}
        \caption{With Rician smoothing}
        \label{fig:histogram_pig6_smooth}
    \end{subfigure}
    \caption{Histograms of error of fit and connection form parameters for pig \# 6}
    \label{histo_pig6}
\end{figure}

Given the association between ADC and our error of frame fit, it is natural to compare these measures quantitatively throughout the myocardium. We did so for the 5 infarcted porcine hearts we analyzed by computing Dice coefficients to describe the overlap, in the following manner:

For the same heart let A be the set of voxels with ADC value > 0.6 and let B be the set of voxels with error of fit $> 15 \degree$. We computed the standard Dice coefficient $\frac{A \cap B}{A \cup B}$ as well as a modified coefficient $\frac{A \cap B}{A}$.
\begin{center}
    \begin{tabular}{|c | c | c |}
         \hline
         Heart & $\frac{A \cap B}{A \cup B}$ & $\frac{A \cap B}{A}$ \\
         \hline
         2 & 0.40 & 0.80 \\ 
         \hline
         4 & 0.43 & 0.89 \\
         \hline
         5 & 0.47 & 0.76 \\
         \hline
         6 & 0.46 & 0.87 \\
         \hline
         7 & 0.27 & 0.94 \\ 
         \hline
    \end{tabular}
\end{center}

These results demonstrate that typically over 80\% of the locations with increased diffusion also yield a high error of fit using our frame fitting method, due to the loss of geometric coherence of fiber orientations. However there are additional locations where fiber orientations are not smooth, typically at the linings of the heart wall, or near the edges of a collapsing and narrow right ventricle. These are picked up by our error of fit measure but not by the ADC measure, likely because there is no increase in collagen there.

\subsection{Histograms of connection forms} \label{histogram_section}

Here are examples of the histograms that we computed on the entire heart for every usable example that we had. The histogram \ref{fig:histogram_pig6_smooth} gives us information on the distribution of $c_{i,j,k}, \forall (i, j, k) \in [1, 3]^3$ and the distribution of the error of fit for each frame axis $\mathbf{f}_1, \mathbf{f}_2$ and $\mathbf{f}_3$.

In this thesis we will focus on the most commonly used parameter for comparison, the helix angle $c_{1,2,3}$. Our tool gives us the mean and the mode (value most represented) of this value through the whole heart. As we have a Gaussian distribution of these values the mode is relevant and helping as it will represent mostly fibers in the core of the heart wall and not take into account (as the mean value does) extreme values that can occur on the boundaries of the heart wall.

The papillary muscles, to name only one side effect, which are mostly aligned in a top-down direction along the long axis, can have non negligible effect on the $c_{1,2,3}$ value and they are difficult to remove from our model as it can be difficult to determine where they start exactly and which layer should then be kept and which should be considered as papillary muscle and therefore removed.

In the table we regroup the results we obtained from all relevant datasets and analyze the total helix angle in degrees that we obtain the following way (where 57.3 is the conversion from rad to $\degree$):
\begin{equation}
    \text{total helix angle} = c_{1,2,3}\cdot \text{\#voxels}\cdot 57.3
\end{equation}
The values that we get are close to the theoretical value of $120\degree$ reported in the literature \cite{piuzephd}.

\begin{center}
    \begin{tabular}{|c | c | c | c| c| c|} 
         \hline
         Heart & \shortstack{Thickness \\ (\# voxels)} & \shortstack{$c_{1,2,3}$ mean \\ rad/voxel} & \shortstack{$c_{1,2,3}$ mode \\ rad/voxel} & \shortstack{Total helix angle \\ (using mean)} & \shortstack{Helix angle \\ (using mode)}\\
         \hline
         2 & 40 & -0.066 & -0.025 & -151.2$\degree$ & -57.3$\degree$ \\ 
         \hline
         4 & 45 & -0.084 & -0.047 & -216.6$\degree$ & -121.2$\degree$ \\
         \hline
         5 & 30 & -0.048 & -0.020 & -82.5$\degree$ & -34.38$\degree$ \\
         \hline
         6 & 30 & -0.069 & -0.053 & -118.6$\degree$ & -91.1$\degree$ \\
         \hline
         7 & 45 & -0.077 & -0.053 & -198.5$\degree$ & -136.6$\degree$ \\ 
         \hline
         25 & 35 & -0.072 & -0.034 & -144.4$\degree$ & -68.2$\degree$ \\
         \hline
    \end{tabular}
\end{center}


\section{Qualitative Results}

\subsection{Importance of the filtering from the fibers...}

In topview figures \ref{fig:pig4_topviews} and a more transverse view where the helix angle is easier to visualize \ref{fig:pig4_helix}, we have a first glimpse at the difference of the fiber orientation and cohesion before and after smoothing the raw data through a Rician filter, which is the appropriate filtering method when dealing with DMRI data.

The region shown in figures \ref{fig:pig4_helix_no_smooth} and \ref{fig:pig4_helix_smooth} is a region away from the infarct where we should see a smooth turning of fibers. This region corresponds to the top-right region of the same slice presented in figures \ref{fig:pig4_topview_no_smooth} and \ref{fig:pig4_topview_smooth}.

Tracing fibers from raw data leads to an extremely noisy data from which it is hard to infer any general structure, which we will see in \ref{fig:histogram_pig6_no_smooth}. Indeed the low magnetic field value (1.5T) is one of the reasons why the data is not of the highest quality, as well as the low number of iterations in the acquisition process.

Thanks to the Rician smoothing it is again possible to guess the helix angle turning from outer wall to inner wall although this is still not perfect, as we want to keep as much information as possible with using the lightest filtering possible.

\begin{figure}[h!]
    \centering
    \begin{subfigure}{.48\textwidth}
        \includegraphics[width=\textwidth]{figures/pig4_topview_no_smooth}
        \caption{Without any smoothing}
        \label{fig:pig4_topview_no_smooth}
    \end{subfigure}
    \begin{subfigure}{.48\textwidth}
        \includegraphics[width=\textwidth]{figures/pig4_topview_smooth}
        \caption{With Rician smoothing}
        \label{fig:pig4_topview_smooth}
    \end{subfigure}
    \caption{Top view of heart fibers for pig 4}
    \label{fig:pig4_topviews}
    \begin{subfigure}{.48\textwidth}
        \includegraphics[width=\textwidth]{figures/pig4_helix_no_smooth}
        \caption{Without any smoothing}
        \label{fig:pig4_helix_no_smooth}
    \end{subfigure}
    \begin{subfigure}{.48\textwidth}
        \includegraphics[width=\textwidth]{figures/pig4_helix_smooth}
        \caption{With Rician smoothing}
        \label{fig:pig4_helix_smooth}
    \end{subfigure}
    \caption{View of the helix angle of heart fibers for pig 4}
    \label{fig:pig4_helix}
\end{figure}

\subsection{...To the tractography and fitting process}

Using the fiber orientation that we can get from the computed tensor matrix, we can run tractography on our results. This process gives us an idea of how fibers should be wrapped around the heart and work together in the muscle structure. Looking at the tractography run on the raw data \ref{fig:pig4_tracto_no_smooth} it is hard to see where the infarct would be by just looking at the tracing of existing fibers and trying to infer their direction. The infarct region, as explained in \cite{wu2007mr}, should be the region with the least coherence in its fiber directions whereas regions remote from the infarct should not be affected in their structure.

\begin{figure}[h!]
    \centering
    \begin{subfigure}{.48\textwidth}
        \includegraphics[width=\textwidth]{figures/pig4_topview_tracto_no_smooth}
        \caption{Without any smoothing}
        \label{fig:pig4_tracto_no_smooth}
    \end{subfigure}
    \begin{subfigure}{.48\textwidth}
        \includegraphics[width=\textwidth]{figures/pig4_topview_tracto_smooth}
        \caption{With Rician smoothing}
        \label{fig:pig4_tracto_smooth}
    \end{subfigure}
    \caption{Tractography of heart fibers for pig 4}
    \label{fig:pig4_tractos}
\end{figure}

Looking at the tractography run on the Rician smoothed data \ref{fig:pig4_tracto_smooth} on the other hand gives a clear hint on the region where the infarct is present (LAD) whereas remote regions do not seem affected by this and wrap smoothly around the heart.

Finally, looking at histograms \ref{fig:histogram_pig6_no_smooth} and \ref{fig:histogram_pig6_smooth} we can get from our procedure to look at statistics on the visualized information presented, we see several clear indications that smoothing will be a capital step in our further analysis of our datasets:
\begin{enumerate}
    \item Shape of Gaussians we get from the datasets. The center is much better defined in for the smoothed data \ref{fig:histogram_pig6_smooth} as well as overall shape which is less spiky
    \item The number of voxels that count in our computations: from $1564$ with raw unfiltered data \ref{fig:histogram_pig6_no_smooth} to $59187$ after the Rician filter \ref{fig:histogram_pig6_smooth}
\end{enumerate}
The number of voxels that count in our computations is the number of voxels where we could converge to a value of connection form matrix $(c_{i, j, k})_{(i, j, k) \in \mathbb{R}^3}$. Voxels where we could not converge to a value after a given number of iterations were discarded in the histogram to limit the noise importance in the final observations.

\subsection{Error of fit vs ADC and FA}

\begin{figure}[h!]
    \centering
    \begin{subfigure}{.31\textwidth}
        \includegraphics[width=\textwidth]{figures/pig2_adc_31}
        \caption{ADC map}
        \label{fig:pig2_adc}
    \end{subfigure}
    \begin{subfigure}{.31\textwidth}
        \includegraphics[width=\textwidth]{figures/pig2_err_31}
        \caption{Error of fit}
        \label{fig:pig2_err}
    \end{subfigure}
    \begin{subfigure}{.31\textwidth}
        \includegraphics[width=\textwidth]{figures/pig2_fa_31}
        \caption{FA map}
        \label{fig:pig2_fa}
    \end{subfigure}
    \caption{Pig 2: Error of fit computed using our framework compared to ADC and FA map}
    \label{fig:pig2}


    \begin{subfigure}{.31\textwidth}
        \includegraphics[width=\textwidth]{figures/pig4_adc_21}
        \caption{ADC map}
        \label{fig:pig4_adc}
    \end{subfigure}
    \begin{subfigure}{.31\textwidth}
        \includegraphics[width=\textwidth]{figures/pig4_err_21}
        \caption{Error of fit}
        \label{fig:pig4_err}
    \end{subfigure}
    \begin{subfigure}{.31\textwidth}
        \includegraphics[width=\textwidth]{figures/pig4_fa_21}
        \caption{FA map}
        \label{fig:pig4_fa}
    \end{subfigure}
    \caption{Pig 4}
    \label{fig:pig4}
    
    \begin{subfigure}{.31\textwidth}
        \includegraphics[width=\textwidth]{figures/pig5_adc_22}
        \caption{ADC map}
        \label{fig:pig5_adc}
    \end{subfigure}
    \begin{subfigure}{.31\textwidth}
        \includegraphics[width=\textwidth]{figures/pig5_err_22}
        \caption{Error of fit}
        \label{fig:pig5_err}
    \end{subfigure}
    \begin{subfigure}{.31\textwidth}
        \includegraphics[width=\textwidth]{figures/pig5_fa_22}
        \caption{FA map}
        \label{fig:pig5_fa}
    \end{subfigure}
    \caption{Pig 5}
    \label{fig:pig5}
\end{figure}
    
\begin{figure}[h!]
    \centering
    \begin{subfigure}{.31\textwidth}
        \includegraphics[width=\textwidth]{figures/pig6_adc_24}
        \caption{ADC map}
        \label{fig:pig6_adc}
    \end{subfigure}
    \begin{subfigure}{.31\textwidth}
        \includegraphics[width=\textwidth]{figures/pig6_err_24}
        \caption{Error of fit}
        \label{fig:pig6_err}
    \end{subfigure}
    \begin{subfigure}{.31\textwidth}
        \includegraphics[width=\textwidth]{figures/pig6_fa_24}
        \caption{FA map}
        \label{fig:pig6_fa}
    \end{subfigure}
    \caption{Pig 6}
    \label{fig:pig6}
    
    \begin{subfigure}{.31\textwidth}
        \includegraphics[width=\textwidth]{figures/pig7_adc_14}
        \caption{ADC map}
        \label{fig:pig7_adc}
    \end{subfigure}
    \begin{subfigure}{.31\textwidth}
        \includegraphics[width=\textwidth]{figures/pig7_err_14}
        \caption{Error of fit}
        \label{fig:pig7_err}
    \end{subfigure}
    \begin{subfigure}{.31\textwidth}
        \includegraphics[width=\textwidth]{figures/pig7_fa_14}
        \caption{FA map}
        \label{fig:pig7_fa}
    \end{subfigure}
    \caption{Pig 7}
    \label{fig:pig7}
    
    \begin{subfigure}{.31\textwidth}
        \includegraphics[width=\textwidth]{figures/pig25_adc_30}
        \caption{ADC map}
        \label{fig:pig25_adc}
    \end{subfigure}
    \begin{subfigure}{.31\textwidth}
        \includegraphics[width=\textwidth]{figures/pig25_err_30}
        \caption{Error of fit}
        \label{fig:pig25_err}
    \end{subfigure}
    \begin{subfigure}{.31\textwidth}
        \includegraphics[width=\textwidth]{figures/pig25_fa_30}
        \caption{FA map}
        \label{fig:pig25_fa}
    \end{subfigure}
    \caption{Pig 25}
    \label{fig:pig25}
\end{figure}

These figures give us a good understanding and overview of how the error of fit can provide supplementary information compared to what the ADC and FA map can give us.

If we focus on the first example \ref{fig:pig2}, we can clearly see in the ADC map \ref{fig:pig2_adc} the region of the infarct where the value is the highest (yellow region); and it really gives the impression of a scar region that contains almost only collagen and no fibrous tissue whatsoever given the apparent anisotropy. But if we look at the FA map \ref{fig:pig2_fa} we can start wondering if there is not a region close to the endocardium where fibers have either resisted or been reconstructed after the infarct in this region.

As for the error of fit, it gives a clear indication of the existence of layers of fibers near the endocardium that still makes the connection with the rest of the heart possible and the contraction, even if badly hindered by the infarct and loss of muscle fibers, still possible.

This arrangement of the surviving fibers as described in the literature overviewed in \ref{surviving_fibers} is also clearly visible in the examples displayed in \ref{fig:pig4} and \ref{fig:pig5}.

The example \ref{fig:pig6} is also a very interesting sample to compare our method to, and it shows that in some cases as mentioned in \cite{ursell1985structural}, intramural muscles seem to be completely absent in this case and can gravely hinder the contraction of the heart.

Our approach seems to be more tolerant in the exhibition of surviving fibers in the infarct region and in some cases show how CR can make the infarct region still a little contractile.

The last provided example \ref{fig:pig25} is used as a control healthy heart to make sure of the extent to which we can use our results. We can clearly see in this healthy example that the error of fit is low pretty much everywhere, except in regions near the heart wall (endothilial and epithilial) and the region where the Left Ventricule (LV) and the Right Ventricule (RV) connect to each other. This is due to difficulty in our method to compute connection forms in areas where fibers tend to go in different directions, depending if they work towards the LV contraction, the RV contraction or else. Then the method performs less well to try to fit smoothly varying geometrical fiber structures.

\subsection{Discussion}

In all the examples seen in the previous section, locations where the ADC value is high are typically also ones where the error of fit is high, with regions of low error of fit being restricted to healthy tissue. In addition though, the error of fit is also consistently high at locations close to the epithelial and endothelial boundaries, signaling a loss in geometric coherence of fibers there. Curiously this phenomena is also seen at the borders of the healthy heart \ref{fig:pig25_err}.

The error of fit shows information about the existence of surviving fibers in the infarct and border zone that can help the contractile function of the heart after the infarct. This is precious knowledge as it is a non-invasive way to qualify the capacity of the heart to keep beating and pumping blood to the body although the heart has suffered from an infarct. Further experiments could be made to support this analysis and show how the efficiency of an infarcted heart is related to the remaining fibers by conducting blood flow experiments before the imaging process.

Another very promising research topic would be to analyze the evolution of surviving fibers at different time frames after infarct, as we were limited to the datasets available which were all 6 weeks after infarct.
