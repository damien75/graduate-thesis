\chapter{Conclusion}

The study of infarcted pig hearts 6 weeks after infarction has allowed us to compare the more mainstream analytical approach that is ADC and FA maps to our error of fit analysis.

We have noticed a difference in the results shown by the scalar approaches (FA and ADC) compared to our geometrical approach that tries to understand and explain how the tissue is organized and how fibers are oriented. As the literature points out, after infarct surviving fiber bundles can persist in the infarct region and have a role in the contraction of the damaged heart. Geometrical reorganization is interesting to observe as it is the cause to tachycardia. Our method offers a good non-invasive approach to try and analyze the problems that can occur after infarct. It shows a good potential as the data is already widely used and therefore analysis can be performed rather easily, without the necessity of a specific experiment to be done. More advanced analytics on the location specific evolution of the connection form parameters could be promising tools to understand heart reconstruction after infarct.

The limitations in this development are the lack of complete and easy to compare histology slices of the hearts we analyzed. This could be a great addition to the conclusions that can be drawn from our error of fit analysis. Another big limitation was the quality of the DMRI data. In spite of our analysis being non-invasive, we have seen in detail how important the quality of the data is and Rician smoothing could not give us better results in every case. High quality in-vivo DMRI data would be a big help to obtain better restuls.

Another interesting study could be performed on infarcted hearts at different stages. As our data was always 6 weeks after infarct, analyzing the evolution of fiber orientations after the infarct could be a meaningful addition to the analysis and conclusions we could draw from the error of fit that we compute. An ideal case would be to have good quality in-vivo MRI images, in order to understand better the effect of the infarct on the heart week by week and how tachycardia manifests itself in the heart wall fiber orientations.