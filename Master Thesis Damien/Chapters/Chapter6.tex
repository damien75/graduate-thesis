\chapter{Conclusion}

The study of infarcted pig hearts 6 weeks after infarction has allowed us to compare the more mainstream analytical approach that is ADC and FA maps to our error of fit analysis.

We have noticed difference in the results shown by the more scalar approaches that are ADC and FA compared to our geometrical approach that tries to understand and explain how the tissue is organized and how fibers are oriented if they still exist. As the literature points out, after infarct surviving fiber bundles can persist in the infarct region and have a role in the contraction of the damaged heart. These geometrical reorganization is interesting to observe as it is the cause to tachycardia.

Our error of fit shows efficiently regions of surviving fibers as well as regions with a high concentration of collagen. A more complete study could be interesting if we had access to infarcted hearts at different stages after infarct to see how the reorganization takes place, and if good quality in-vivo MRI images could be obtained then a much better understanding of the effect of the infarct and the cause of tachycardia could be achieved.