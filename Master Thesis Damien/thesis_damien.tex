\documentclass[12pt,Bold,letterpaper,TexShade]{mcgilletdclass}
\usepackage{graphicx}
\usepackage{caption}
\usepackage{subcaption}
\usepackage[dvips]{geometry}
\usepackage{mathtools}
\usepackage{algorithm}
\usepackage{algorithmic}
\addtolength{\hoffset}{0pt}                        
\addtolength{\voffset}{0pt}                        
\usepackage{amsthm}
\usepackage{rotating}
\usepackage{adjustbox}
\usepackage{longtable}
\usepackage{indentfirst}
\usepackage{amsfonts}
\usepackage{amsmath}







\SetTitle{\huge{Cartan Frame Analysis of Hearts with Infarcts}}%
\SetAuthor{Damien Goblot}%
\SetDegreeType{Master of Computer Science}%
\SetDepartment{School of Computer Science}%
\SetUniversity{McGill University}%
\SetUniversityAddr{Montr\'eal,Quebec}%
\SetThesisDate{2016-08-13}%
\SetRequirements{A thesis submitted to McGill University in partial fulfillment of the \\requirements for the degree of Master of Science in Computer Science}%
\SetCopyright{\copyright\ \the\year\ Damien Goblot}%

\makeindex[keylist]
\makeindex[abbr]

\listfiles%
\begin{document}

\maketitle%

\begin{romanPagenumber}{2}%

\SetDedicationName{\MakeUppercase{Dedication}}%
\SetDedicationText{\include*{Dedication/dedication_page}}%
\Dedication%

\SetAcknowledgeName{\MakeUppercase{Acknowledgements}}%
\SetAcknowledgeText{\include*{Acknowledgement/acknowledgment_page}}%
\Acknowledge%


%%%%%%%%%%%%%%%%%%%%%%%%%%%%%%%%%%%%%%%%%%%%%%%%%%%%%
%%         English Abstract                        %%
%%%%%%%%%%%%%%%%%%%%%%%%%%%%%%%%%%%%%%%%%%%%%%%%%%%%%
\SetAbstractEnName{\MakeUppercase{Abstract}}%
\SetAbstractEnText{ This thesis considers a method for computing skeletal representations based on the average outward flux (AOF) of the gradient of the Euclidean distance function to the boundary of a 2D object through the boundary of a region that is shrunk. It then shows how the original method, developed by Dimitrov \textit{et al.}  can be optimized and made more efficient and proposes an algorithm for computing flux invariants which is a number of times faster. It further exploits a relationship between the AOF and the object angle at endpoints, branch points and regular points of the skeleton to obtain more complete boundary reconstruction results than those demonstrated in prior work. Using this optimized implementation, new measures for skeletal simplification are proposed based on two criteria: the uniqueness of an inscribed disk as a tool for defining salience, and the limiting average outward flux value. The simplified skeleton when abstracted as a directed graph is shown to be far less complex than popular skeletal graphs in the literature, such as the shock graph, by a number of graph complexity measures including: number of nodes, number of edges, depth of the graph, number of skeletal points, and the sum of topological signature vector (TSV) values. We conclude the thesis by applying the simplified graph to a view-based object recognition experiment previously arranged for shock graphs. The results suggest that our new simplified graph yields recognition scores very close to those obtained using shock graphs but with a smaller number of nodes, edges, and skeletal points.  }
\AbstractEn%

%%%%%%%%%%%%%%%%%%%%%%%%%%%%%%%%%%%%%%%%%%%%%%%%%%%%%
%%         French Abstract                         %%
%%%%%%%%%%%%%%%%%%%%%%%%%%%%%%%%%%%%%%%%%%%%%%%%%%%%%
\SetAbstractFrName{\MakeUppercase{ABR\'{E}G\'{E}}}%
\SetAbstractFrText{Ce m\'{e}moire propose une m\'{e}thode pour calculer des repr\'{e}sentations squelettiques en fonction du flux moyen d\'{e}crit par le gradient de la fonction de distance Euclidienne aux limites d'un objet 2D qui r\'{e}tr\'{e}cit.  La m\'{e}thode originale d\'{e}velopp\'{e}e par Dimitrov \textit{et al.}  est ensuite optimis\'{e}e afin de calculer des invariants de flux plus rapidement. Une relation entre l'AOF et l'angle de l'objet aux extr\'{e}mit\'{e}s (aux points de branches et des points r\'{e}guliers du squelette) est exploit\'{e}e afin d'obtenir une reconstruction plus pr\'{e}cises des limites de l'objet par rapport aux travaux pr\'{e}c\'{e}dents. En utilisant cette impl\'{e}mentation optimis\'{e}e, de nouvelles mesures de simplification de squelettes sont propos\'{e}es selon deux crit\`{e}res: l'unicit\'{e} d'un disque inscrit comme un outil permettant de d\'{e}finir la saillance, et la limitation de la moyenne du flux \'{a} l'ext\'{e}rieur. Il est d\'{e}montr\'{e} que le squelette simplifi\'{e}, abstrait par un graphe orient\'{e}, est beaucoup moins complexe que des graphes squelettiques conventionnels mentionn\'{e}s dans la litt\'{e}rature, tel que le graphe de choc. Les mesures de complexit\'{e} de graphe comprennent le nombre de nœuds, le nombre de bords, la profondeur du graphe, le nombre de points du squelette et la somme des valeurs du vecteur des signes topologiques (TSV). La th\`{e}se se finit en appliquant le graphe simplifi\'{e} au probl\`{e}me de reconnaissance d'objets bas\'{e}e sur la vue, pr\'{e}alablement adapt\'{e} pour l'utilisation de graphes de choc. Les r\'{e}sultats sugg\`{e}rent que notre nouveau graphe simplifi\'{e} atteint des performances similaires \`{a} celles des graphes de choc, mais avec moins de nœuds, de bords et de points du squelette plus rapide.}%
\AbstractFr%




\TOCHeading{\MakeUppercase{Table of Contents}}%
\LOTHeading{\MakeUppercase{List of Tables}}%
\LOFHeading{\MakeUppercase{List of Figures}}%
\tableofcontents %
\listoftables %
\listoffigures %

\end{romanPagenumber}

%\mainmatter %
 
\chapter{Introduction}

The technological improvement of noninvasive medical imaging techniques in the past couple of decades has allowed us to get a better understanding of the heart's structure \cite{sinusas2008multimodality}. This is also thanks to the increasing amount of data that can be collected more and more easily and with better definition as time goes by. \\
As the imaging techniques get better \cite{shaw2010cardiovascular}, the analysis of the images we get still plays a major role in understanding the causes and effects of the disease and has to be used in a clever way to drive further down the number of cardiovascular diseases that is still the leading global cause of death \cite{mozaffarian2015heart} since 1900 (except for the year 1918). \\
Although the variety of imaging modalities is very wide and information can be convened from these multiple sources, the most valuable for a clinician will be a fast processed and easy to understand and visualize data that we hope our imaging tools of fibers can provide. The visualization of those in 3D is key since they are the pillar of the mechanical behavior of the heart \cite{hooks2002cardiac}.\\
The core organization of these fibers can be altered by various common pathologies and lead to a deficient contraction of the heart \cite{beg2004computational}. \\
Numerous former histological studies have shown how fibers in a healthy heart are smoothly wrapped around it and that the variation of the helix angle (measure of fiber orientation) is a common descriptor of local fiber geometry \cite{geerts2002characterization}. Computational aided visualization of the disorganization that can occur in the heart structure could be helpful for modern treatments that feature tissue engineering methods to restore the contractile properties of the heart \cite{caplan2006mesenchymal,laflamme2007cardiomyocytes,laflamme2005regenerating, song2012heart, zimmermann2004engineered}, or in more extreme cases where they proceed with reconstruction, or even ablation of infarcted regions \cite{athanasuleas2004surgical, di2001effects, jones2009coronary, sartipy2005dor}.

\section{Objectives of this Thesis}

In this thesis we start from our knowledge of Maurer-Cartan connection form description of the fibers of the heart \cite{pami2015}. From this basis we try and analyze the possible impacts that an infarct can have on the fiber orientation in the infarct area and areas away from the infarct regions.
\chapter{Literature Review}

\section{Cardiac Diffusion MRI}

Magnetic Resonance Imaging (MRI) is the imaging method that was used to acquire the data on which our results are based. It is non-invasive and especially adapted to soft-tissues. \cite{bakermans2008} We will explain briefly how it works and what information it gives us.

\subsection{Physics of nucleus magnetic moment}

Magnetic Resonance acquisitions rely on the physical properties of the body, and more specifically the most abundant nucleus which is hydrogen (H). Indeed water molecules ($H_2O$, therefore 2 nucleus of hydrogen in each molecule) represent more than 60\% of the entire body composition. This hydrogen nucleus H consists of 1 proton and 1 electron of opposite electrical charge. The proton spins around itself and therefore creates a magnetic moment which makes the nucleus behave like a magnet to the outside world.
\begin{figure}
    \centering
    \begin{subfigure}{.28\textwidth}
        \includegraphics[width=\textwidth]{figures/spin_cm}
        \caption{The hydrogen proton behaves like a magnetic dipole - Adapted from \cite{dipole}}
        \label{fig:hydrogen_magnet}
    \end{subfigure}
    \hfill
    \begin{subfigure}{.68\textwidth}
        \includegraphics[width=\textwidth]{figures/Precession_Larmor_2}
        \caption{The Larmor precession when an exterior magnetic field is present - \cite{hydrogen}}
        \label{fig:larmor_precession}
    \end{subfigure}
    \caption{Magnetic behavior of the hydrogen nucleus}
    \label{fig:hydrogen}
\end{figure}

Applying an external magnetic field $\mathbf{B_0}$ leads to a precession around the direction of $\mathbf{B_0}$ of the nuclear magnetic moment of the nucleus. Its magnetic moment will have a characteristic Larmor frequency $\omega_0$. All the magnetic moments of the nuclei in a specific region will average out to a net magnetization vector $\mathbf{M_0}$ aligned with $\mathbf{B_0}$. However, it is important to note that all protons will have a different phase and that the precession is asynchronous between all protons, as shown in \ref{fig:larmor_precession}.

Then the idea is to apply a secondary and much less intense magnetic field $\mathbf{B_1}$, with $||\mathbf{B_1}|| \simeq 10^{-6}||\mathbf{B_0}||$ in a direction in the plane orthogonal to the direction of $\mathbf{B_0}$, rotating around the direction of $\mathbf{B_0}$ with the Larmor frequency $\omega_0$ to bring the nuclei to resonance and out of their equilibrium, ie into an excited state, as shown in Fig. \ref{fig:spin_excitation}. A torque will indeed be applied to the magnetic moment of the nucleus, coming from the Lorentz force: $\tau = \mathbf{M} \times \mathbf{B_1}$ where $\mathbf{M}$ is the magnetic moment of the nucleus. This can be achieved by transmitting radio frequency (RF) pulses. Then the motion of all nucleii will be synchronized and will be in phase.

\begin{figure}
    \centering
    \includegraphics[width=.5\textwidth]{figures/spin_excitation}
    \caption{Torque effect on the magnetic moment of the proton $\mathbf{M}$ when applying an extra magnetic field $\mathbf{B_1}$}
    \label{fig:spin_excitation}
\end{figure}

Once the impulse stops, the protons will return to their non-excited states and in that process will emit another magnetic signal (weak compared to $\mathbf{B_0}$) that will be analyzed, since it is specific to each proton. This information will be used to understand the distribution of the orientations of the magnetic moments in the environment.

\subsection{Using Magnetic Resonance for Imaging}

Once a nucleus is in its excited state, its relaxation as illustrated in Fig. \ref{fig:spin_relaxation} and the outcome magnetization vector in the $(x, y)$ plane $\mathbf{M_{x, y}}$ induces a current recorded by a RF receiving coil.
\begin{figure}
    \centering
    \includegraphics[width=.5\textwidth]{figures/spin_relaxation}
    \caption{Relaxation of the magnetic spin after the magnetic pulse $\mathbf{B_1}$ stops}
    \label{fig:spin_relaxation}
\end{figure}

This is the signal that is measured in a MR experiment and will be used to acquire the data.

For acquisitions in 2D or 3D, gradients are applied to the RF pulse(so that the excitation of molecules is characteristic to a location. As a result, the receiving coil can differentiate the localization of the signal more easily depending on this modulation. Typically in 2D frequency and phase encoding is used for the 2 degrees of freedom. In 3D slice selection is accomplished via a linear magnetic gradient, so that the main exterior magnetic field $\mathbf{B_0}(z)$ is different for each slice and leads to different Larmor values for each value of $z$.

In our case all MRI are acquired ex-vivo, which means that the acquisition process is more simple as the subject is perfectly still and several acquisitions can be performed without having to worry about the subject's motion. In the presence of subject motion more elaborate methods have to be applied like FLASH imaging.

Standard MRI is based mainly on the specific relaxation properties of water molecules, and this allows the physician to get images that represent efficiently the different types of tissues depending on their concentration in water molecules - highest in water and fat specifically. In order to focus on the tissue organization Diffusion MRI takes its place in giving out local characteristics of molecular diffusion. This time 2 pulses are applied and the motion of water molecules between these 2 sequences will be exhibited.

\section{Diffusion Tensor Imaging}

Magnetic resonance diffusion tensor imaging (DTI) is a method based on MRI and the diffusion mechanism of water molecules to assess the direction of fibers from the anisotropic property of the diffusion of these molecules in an oriented or elongated structure.

\subsection{Physical diffusion property of molecules}

In general the DTI signal is based on the proton of the hydrogen nucleus present in water molecules.

In a non-restrictive volume, the motion of molecules is Brownian and has a Gaussian distribution at equilibrium that is linked to the temperature of the environment and other properties. This Gaussian distribution will be affected by the presence of physical barriers that will decrease the diffusivity. Molecular motion will in fact be hindered in the direction perpendicular to the obstacles. When barriers are present the diffusion will change from isotropic to anisotropic. If we look at a free water molecule, its diffusion would have the shape of a sphere whereas with boundaries in the medium the diffusion of the same water molecule would be more of an ellipsoid with principal direction orthogonal to the boundaries.

\subsection{Diffusion weighted imaging} \label{dw_imaging}

Two short and intense successive pulsed field gradients (PFGs) are applied to the region of interest. The first makes protons have a precession phase proportional to their position along the direction of the gradient. Then a second one is applied with the exact opposite direction so that molecules that didn't move will have a null precession phase whereas those that moved will have a precession phase proportional to their motion in the direction of the PFG. Here we refer single water molecules but in reality we are considering a distribution of water molecules. For unbounded water molecules the distribution will be centered around a null motion whereas water molecules bounded by a rectangular container for instance will have a motion distribution centered around the direction of the longest vertex.
\begin{figure}
    \centering
    \includegraphics[width=.5\textwidth]{figures/pgse}
    \caption{Pulse Gradient Spin Echo (PGSE) sequence}
    \label{fig:pgse}
\end{figure}

The acquisition sequence, also called a Pulse Gradient Spin Echo (PGSE) sequence, can be described as follows: \cite{assemlal2011recent}
\begin{enumerate}
    \item A first PFG is applied in a direction orthogonal to the main magnetic field $\mathbf{B_0}$ for $t \in [0, \delta]$.\\
    This will have the effect of flipping the nuclear spins in the direction of the impulse
    \item Once the impulse is over, the magnetic moment of the nuclei will start dephasing as they will start rotating around the direction of $\mathbf{B_0}$, each of them with their specific frequency related to the strength of the magnetic field at their specific location
    \item After a diffusion time $t = \Delta$, another PFG is applied but this time with a direction opposite to the first impulse for $t \in [\Delta, \Delta + \delta]$, which will refocus the spin phases
    \item Finally at the echo time $t = TE$, the diffusion signal is received by the scanner coil with a loss proportional to the displacement of water molecules during this process. It is the information given by the displacement that is related to the micro structure of the studied environment.
\end{enumerate}

The equation of Nuclear Magnetic Resonance (NMR) was given by Stejskal and Tanner in 1965: it quantifies the decrease in signal intensity and relates this to the apparent diffusion coefficient $\mathbf{D}$.\\
Let $\mathbf{G}(t) = \Big(H\big(t - t_1\big) - H\big(t - (t_1 + \delta)\big)\Big)\mathbf{u} + \Big(H\big(t - t_2\big) - H\big(t - (t_2 + \delta)\big)\Big) \mathbf{v}$\\
where $\mathbf{u}, \mathbf{v}$ are unit vectors in the direction of the applied gradients, and $H(t)$ the Heaviside function at a given time $t$. This function is in fact the mathematical description of the PGSE sequence. Then we get the following Stejskal and Tanner equation \cite{Stejskaltanner}:

\begin{equation} \label{eq:tanner}
\frac{S(TE)}{S_0} = \exp\Big({-\gamma^2G^2\delta^2(\Delta - \frac{\delta}{3})\mathbf{G}^T\mathbf{D}\mathbf{G}\Big)} = \exp (-b \times \mathbf{D})
\end{equation}
where:

\begin{itemize}
    \item $\gamma$: gyromagnetic ratio, constant specific to each molecule
    \item $G$: PFG strength
    \item $\delta$: duration of the pulse
    \item $\Delta$: time between 2 pulses
    \item $\mathbf{D}$: apparent diffusion coefficient
    \item $b$ factor suggested by LeBihan \cite{lebihan1985imagerie}: $b = \gamma^2G^2\delta^2(\Delta - \frac{\delta}{3})||\mathbf{G}||^2$
\end{itemize}

\subsection{To get a diffusion tensor matrix} \label{diffusion_tensor_matrix}

Diffusion anisotropy can be described mathematically by a 3x3 diffusion tensor:
\begin{equation}
    \mathbb{D} = \begin{pmatrix}
    D_{xx} & D_{xy} & D_{xz} \\
    D_{yx} & D_{yy} & D_{yz} \\
    D_{zx} & D_{zy} & D_{zz}
    \end{pmatrix}
\end{equation}
with $D_{ij}$ the apparent diffusion coefficient (ADC) in direction $(i, j)$. In an environment dominated by water molecules, principles of thermodynamics dictate that $\mathbb{D}$ is symmetric, which means that $D_{ij} = D_{ji}$.

In conclusion we need at least 6 independent gradient directions to determine all 6 degrees of freedom in this matrix. In practice 7 directions are used to get more accurate results, along with a non-weighted measurement for normalization.

The values that we analyze most frequently are ADC and fracional anisotropy (FA). The ADC is an interesting measure as it can show difference in average diffusion values from one region to another - it was particularly adopted to assess the severity of injury in adult stroke patients. It has been challenged recently as it is extremely sensitive to tissue microstructure. As for the FA, it is a very interesting measurement that reflects fiber density and diameter.
\begin{equation}
    ADC = \frac{tr{\mathbb{D}}}{3} = \frac{\lambda_1 + \lambda_2 + \lambda_3}{3}
\end{equation}
\begin{equation}
    FA = \sqrt{\frac{(\lambda_1 - \lambda_2)^2 + (\lambda_2 - \lambda_3)^2 + (\lambda_3 - \lambda_1)^2}{\lambda_1^2 + \lambda_2^2 + \lambda_3^2}}
\end{equation}

\section{Heart fiber geometry}

From the 1970s, papers have shown how myofibers in the left ventricle (LV) are aligned in a way that they form helical curves \cite{savadjiev2012heart}. This alignment has proved to be optimal for its mechanical function. Heart fibers stay locally packed together and almost parallel to each other, although they are bundled into a special surface: a generalized helicoid \cite{blair1978generalization}. This is a minimal surface that maintains this property through the whole beat cycle. The bundle of myofibers in generalized helicoids is their characteristic orientation through the heart wall.

\subsection{The Generalized Helicoid Model (GHM)}

We will introduce a local frame field $\mathbf{f}_1,\mathbf{f}_2,\mathbf{f}_3 : \mathbb{R}^3 \to \mathbb{R}^3$ that will allow us to describre more easily the GHM. In this description, we follow the same local frame field definition as chosen in previous publications \cite{pami2015, savadjiev2012heart}:
\begin{itemize}
    \item $\mathbf{f}_1$ is the local orientation of the fiber
    \item $\mathbf{f}_3$ is the local heart wall normal
    \item $\mathbf{f}_2 = \frac{\mathbf{f}_3 \times \mathbf{f}_1}{||\mathbf{f}_3 \times \mathbf{f}_1||}$: to get a right-handed orthogonal coordinate frame field
\end{itemize}

\begin{figure}
    \centering
    \includegraphics[width=.5\textwidth]{figures/frame_fields}
    \caption{Examples of possible local frame fields in a world coordinate system $(x, y, z)$}
    \label{fig:frame_fields}
\end{figure}

In the GHM model the fiber orientation is expressed at each point $(x, y, z) \in \mathbb{R}^3$ by the function:
\begin{equation}
\theta (x, y, z) = \arctan \Bigg( \frac{K_Tx + K_Ny}{1 + K_Nx - K_Ty} \Bigg) + K_Bz
\end{equation}
where $\theta$ is the angle $\angle (\overrightarrow{x}, \overrightarrow{y})$
\begin{itemize}
    \item $K_T$ causes bending in the tangent direction to the fiber
    \item $K_N$ makes fibers fan out, away from each other, or towards each other, depending on its sign, in the $\mathbf{f}_2$ direction
    \item $K_B$ copies rotated versions of the fiber in planes parallel to the original $(x, y)$ plane, i.e. in the direction orthogonal to the heart wall
\end{itemize}
In general those values describe the amount of change in fiber orientation when going in each of the direction: tangential, orthogonal and out of plane.

Analyses of these parameters on different species has shown on the entire heart volume \cite{savadjiev2012heart} that the GHM fits with small values of $K_T$ and $K_N$ and high values of $K_B$. This validates the helicoidal model approach and the characteristic of the helix angle in heart myofiber structure.

From this analysis of helix angle using a model based description (GHM), by that we mean a fixed reference frame to measure parameters we are going to another approach that is model free. The method of moving frames we will introduce in the next section allows us to have a reference frame that can be redefined for each fiber we are focusing on, and therefore is free of this constraint of a general model based description.

\subsection{The Maurer-Cartan form}

The Maurer-Cartan form operator has proved useful in measuring the differential structure of a manifold, and the theory of moving frame has shown good results in the modeling of the heart fiber geometry \cite{pami2015}. The main idea is to apply the theory of moving frames but reverse it: tuning the Maurer-Cartan connection forms can enable us to generate manifolds based on assumptions on the structure that represent in the best way possible the actual heart fiber structure. This approach can be executed by using different fitting methods.

\subsubsection{Theory of moving frames}

We utilize the framework described in \cite{de1990ventricular} to describe the geometry of fiber orientation in the heart wall via rotations of a frame field that is fit to the diffusion MRI data. 

Let a point $\mathbf{x} = \sum_i{x_i\mathbf{e_i}} \in \mathbb{R}^3$ be expressed in terms of $(\mathbf{e_1}, \mathbf{e_2}, \mathbf{e_3})$, the natural basis for $\mathbb{R}^3$.\\
We define a right-handed orthonormal frame field $\mathbf{f}_1,\mathbf{f}_2,\mathbf{f}_3 : \mathbb{R}^3 \to \mathbb{R}^3$.\\
Each frame axis can be expressed by the rigid rotation $f_i = \sum_i{a_{i,j}\mathbf{e}_j}$, where $\mathbf{A} = {a_{i,j}} \in \mathbb{R}^{3 \times 3}$ is a differentiable attitude matrix such that $\mathbf{A}^{-1} = \mathbf{A}^T$.\\
Treating $\mathbf{f}_i$ and $\mathbf{e}_j$ as symbols, we can write:
\begin{equation}
\begin{bmatrix}
    \mathbf{f}_1 \\
    \mathbf{f}_2 \\
    \mathbf{f}_3
\end{bmatrix} = \mathbf{A} \times \begin{bmatrix}
    \mathbf{e}_1 \\
    \mathbf{e}_2 \\
    \mathbf{e}_3
\end{bmatrix}
\end{equation}
Since each $\mathbf{e}_i$ is constant, the differential geometry of the frame field is completely characterized by $\mathbf{A}$. Taking the exterior derivative on both sides, we have:

\begin{align} \label{eq:1}
    \partial \begin{bmatrix}
                \mathbf{f}_1 \\
                \mathbf{f}_2 \\
                \mathbf{f}_3
            \end{bmatrix} &= (\partial \mathbf{A})\mathbf{A}^{-1}  \begin{bmatrix}
                \mathbf{f}_1 \\
                \mathbf{f}_2 \\
                \mathbf{f}_3
            \end{bmatrix} \\
            &= \mathbf{C}
            \begin{bmatrix}
                \mathbf{f}_1 \\
                \mathbf{f}_2 \\
                \mathbf{f}_3
            \end{bmatrix}
\end{align}
where $\partial$ denotes the exterior derivative, and $\mathbf{C} = (\partial \mathbf{A}) \mathbf{A}^{-1} = (c_{i,j}) \in \mathbb{R}^{3 \times 3}$ is the Maurer-Cartan matrix of connection forms $(c_{i,j})$.\\
Writing $\mathbf{f}_i$ as symbols, Eq. \ref{eq:1} is to be understood as $\partial \mathbf{f}_i = \sum_j{c_{i,j}\mathbf{f}_j}$.

The Maurer-Cartan matrix is skew symmetric with zeros as diagonal entries so there are at most 3 independent, non-zero 1-forms: $c_{1,2}$, $c_{1,3}$, and $c_{2,3}$.

\begin{equation}
    \mathbf{C} = \begin{pmatrix}
    0 & c_{1,2} & c_{1,3} \\
    -c_{1,2} & 0 & c_{2,3} \\
    -c_{1,3} & -c_{2,3} & 0
    \end{pmatrix}
\end{equation}
1-forms operate on tangent vectors through contraction, written as $\partial \omega \langle \boldsymbol{\upsilon} \rangle \in \mathbb{R}$ for a general 1-form $\partial \omega = \sum_i{\omega_i \mathbf{e_i}}$ and tangent vector $\boldsymbol{\upsilon} \in \mathbb{R}^3$, which yields:

\begin{align}
    \partial \omega \langle \boldsymbol{\upsilon} \rangle &= \sum_i{\omega_i \partial \mathbf{e_i}} \big{\langle} \sum_j{\upsilon_j \mathbf{e_j}} \big{\rangle} \\
    &= \sum_i{\omega_i \upsilon_i}
\end{align}
since
\begin{equation}
    \partial \mathbf{e}_i \langle \mathbf{e}_j \rangle = \delta_{i,j}
\end{equation}
where $\delta_{i,j}$ is the Kronecker delta.
It turns out that the space of linear models for smoothly varying frame fields is parametrized by the 1-forms $c_{i,j}$. Since only 3 unique non-zero combinations of $c_{i,j}$ are possible, there are in total 9 connections $c_{i,j,k}$ that fully characterize the local frame field geometry.

\subsubsection{Tuning connection form parameters by energy minimization} \label{fitting_method}

\begin{figure}
    \centering
    \includegraphics[width=.5\textwidth]{figures/fibers_neighborhood3d}
    \caption{Example of a fiber neighborhood of 27: the center fiber (dark blue) is where we will have a first guess for the values of the connection forms $c_{i,j,k}$ and then will run an energy minimization algorithm to come up with parameters that give the same orientation for the neighbors as we can see in the ground truth here}
    \label{fig:fibers_neighborhood3d}
\end{figure}

Finding out the correct values for the connection form parameters $c_{i,j,k}(\mathbf{x}), \forall \mathbf{x} \in \mathbb{R}^3$ is done by energy minimization within a neighborhood $\Omega$, as we can see in Fig. \ref{fig:fibers_neighborhood3d}. This approach makes sense as the GHM is still dependent on the location in the heart, indeed the rate of turning will not be the same close to the apex for instance or where the heart wall is thinner.\\
Therefore the optimal parameters are defined as follows:
\begin{align*}
    c^*_{i,j,k}(\mathbf{x}) &= \text{argmin}_{c_{i,j,k}} \frac{1}{\Omega} \sum_{\mathbf{v} \in \Omega} \sum_{i = 1}^3 \epsilon_i(\mathbf{x} + \mathbf{v}) \\
    \epsilon_i(\mathbf{x} + \mathbf{v}) &= \arccos \Big( \mathbf{f}_i(\mathbf{x} + \mathbf{v}) . \mathbf{\tilde{f}}_i(\mathbf{x} + \mathbf{v}) \Big)
\end{align*}
where:
\begin{itemize}
    \item $\epsilon_i$ is the error function for each axis 
    \item $\mathbf{f}_i$ is the actual frame orientation at this location
    \item $\mathbf{\tilde{f}}_i$ is the current approximation of the frame orientation
\end{itemize}
The optimization strategy can be one of the standard well-known algorithms for these specific tasks: either Nelder-Mead \cite{press2007numerical} or BOBYQA \cite{powell2009bobyqa}.

\subsection{Combining both for a good description of heart fiber geometry}

The heart is made of elongated muscle cells called cardiomyocytes that are organized within a collagen matrix. The macroscopic shape is one of a truncated ellipsoid, while on a microscopic scale myocytes are stacked and form myofibers \cite{streeter1969fiber}.

Work on histology has shown how cardiac myofibers wrap around the ventricles in helical curves \cite{buckberg2002basic}. The ubiquitous analysis of myofiber geometry states out that the helix angle - $\theta$ or $c_{1,2,3}$ in previous sections, or the angle $\angle (\overrightarrow{\mathbf{f}^{i}_1}, \overrightarrow{\mathbf{f}^{i+1}_2})$ where $i$ is a tangent plane to the heart wall at this location - varies smoothly in Fig. \ref{fig:helix_angle} with a total change in Fig. \ref{fig:total_angle} of around 120° for mammals.
\begin{figure}
    \centering
    \includegraphics[width=.78\textwidth]{figures/total_angle}
    \caption[b]{Visualization of the helix angle with a transmural point of view}
    \label{fig:total_angle}
\end{figure}
\begin{figure}
    \includegraphics[width=.78\textwidth]{figures/helix_angle}
    \caption{The helix angle varies smoothly from one tangent plane to the other from the epicardium to the endocardium}
    \label{fig:helix_angle}
\end{figure}

Applying this method to the heart can be done by choosing $\mathbf{f}_i$ as described previously in the GHM model. $\mathbf{f}_1$ is easy to set from the DTI data, while $\mathbf{f}_3$ is determined locally by an approximation of the transmural direction at each location using a mask of the studied heart and a euclidian distance computed between the epicardium (outtermost layer of the heart wall) and endocardium (innermost layer) of the heart.

The Maurer-Cartan connection form approach is a good one to describe the cardiac frame field, where the 3 main geometrical features are exhibited by parameters of this model:
\begin{itemize}
    \item $c_{1,2,3}$ for the helix angle, or trans-mural penetration of the heart wall
    \item $c_{1,3,1}$ quantifies the short-axis (in plane for a given slice) curvature of the heart wall
    \item $c_{2,3,2}$ represents the long-axis (from one slice to another) curvature
\end{itemize}
As a reminder, in general, these coefficients $c_{i,j,k}$ express the rate of turn of the frame vector $\mathbf{f}_i$ towards $\mathbf{f}_j$ when $\mathbf{x}$ moves in the direction $\mathbf{f}_k$.

In previous sections we reviewed MRI and DMRI as they are the technologies involved in the acquisition of our data. Then a review of the GHM and the Maurer-Cartan forms helped us introduce the tools we rely on for the geometric characterization of fiber orientation. In the following section we provide the reader a brief review of infarcts and their effects, as it is the main application area of my thesis. This subject is particularly interesting as not much is yet known about how fibers are remodeled in the presence of infarcts.

\section{Infarcts and their impact on heart fiber geometry}

Studies have been trying to determine the consequences of myocardial infarctions in small (rat \cite{weisman1985global}) hearts and larger (porcine \cite{wu2007mr, mediamihaela}) hearts. In these studies it was shown that acute infarction led to a decrease in regional wall thickness and an increased radius of curvature, mostly in the border zone (BZ) - which is the zone close to the dense scar region but with a more heterogeneous nature due to the persistence of surviving blood vessels that keep on supplying oxygen to isolated cells.

\subsection{Myocardial remodeling after infarct}

After an infarct occurs, a remodeling happens in the heart, which means that the normally existing structure goes through a rearrangement process as a result of the infarct \cite{swynghedauw1999molecular}. Cardiac Remodeling (CR) is determined by the following facts:
\begin{enumerate}
    \item Adaptation process of the myocytes and collagen to new geometrical conditions
    \item Fibrosis - increased collagen concentration, due to senescence - deterioration with age, ischemia, inflammatory processes
    \item cell death as a result of fibrosis, the major marker of cardiac failure
\end{enumerate}
One of the first consequences of CR is a decrease of the wall thickness, which leads immediately to an increased volume of the Left Ventricule (LV). Another direct effet is cardiac heart hypertrophy, with a multiplication of nonmuscle cells (fibroblasts and endothelial cells) in the place of former muscle myocytes. A study on rats \cite{mccormick1994regional} showed that the hypertrophy lead to a 25\% increase of the heart mass 13 weeks after infarct.

Myocardial fibrosis is one of the major sign of malignancy in the CR process, which can lead to death by cardiac failure (CF) or severe arrhythmias.

\subsection{Effects of wall stress on tissue composition and geometry}

A main problem in the study of the consequences of an infarct on a mammal heart is the important cost that makes it difficult to get various data on animal heart after infarct, as there are limited means,  and was for a long time even impossible, to analyzing it in-vivo \cite{holmes1994scar}. Several experiments still have been made and the following chronological evolution of the cardiac fiber geometry emerged:
\begin{itemize}
    \item Local infarct expansion 1 week after the initial infarct, characterized by a stretch in a plane tangent to the epicardium and wall thinning.
    \item Scar shrinkage 3 weeks after infarction. The CR, even without reestablishing a contractile function as efficient as before the infarct, works around the infarct to compensate the work overload on the scar region.
    \item Alignment of myocardial collagen fibers with the direction of the greatest applied stress, which takes place during infarct healing
\end{itemize}

\subsection{Characteristics of collagen in infarcted hearts and arrangement of the surviving fibers} \label{surviving_fibers}

Studies \cite{mccormick1994regional} have shown that after infarct the whole heart mass increase by an average of 25\%. The infarcts were affecting the LV, and the difference in mass goes up to 58\% if only the LV (ventricule affected by the infarct) is considered.

A major impact of an infarct on the heart is on the collagen concentration that goes up significantly both in the scar tissue, the BZ which is specifically rich in fibrosis and in more distant regions. The concentration of collagen can double in the most affected regions of the heart. In general and throughout the whole heart the collagen concentration increases dramatically and is paired with more cross-linking of these cells.

Studies on human hearts \cite{de1990ventricular} have shown that arrhythmias and tachycardia can be caused by abnormal membrane variables and abnormal geometrical arrangement of the myocardial fibers.

Transmembrane potentials were taken and shown little difference between patients with ventricular tachycardia in a chronic phase of myocardial infarction and normal patients. This clearly indicates that the surviving myocardial fibers within an infarct can come back close to normal after a CR process. It has also been observed that surviving bundles in the subendocardial layers as well, and more surprisingly, as intramurally and subepicardially. These bundles of surviving myocytes showed an anisotropy similar to the papillary muscle of the heart, and therefore a similar coupling resistance, which makes sense as the activation signal propagates itself like a zigzag through the infarcted region of the heart.

After a transmural myocardial infarction, the outtermost epicardial muscle survives and its structure contributes greatly in the occurence of the reentry. These surviving muscle cells in the BZ of the epicardium are like a thin sheet over the infarct. The connections with other rare intramural muscle is not present very often, and this means that the impulses are only 2-dimensional as they cannot come from below, which prevents an appropriate circus movement and can lead to tachycardia \cite{ursell1985structural}.
% Point out that for the first time in the literature Cartan's moving frame method is used to assess damage in infarcted hearts by looking at the degree to which fiber orientations in a local neighborhood of a voxel are consistent with its 1-form characterization. Point out that this parametric approach also leads to a geometric model for fibers in healthy regions. Point out that regions of high error explain more than 80\% of those regions with high ADC. However, they also include regions near the heart wall (endothilial and epithilial) linings. Talk about the promise of Cartan frame fitting error in the early stages post infart, not just after 4-6 week period which has been our current focus.

\chapter{Experimental Setup}

The data on which our results are based are all porcine hearts that were gratefully provided to us by Mihaela Pop from the Sunnybrooke University.

\section{Data acquisition of pig infarcted and healthy hearts}

Each porcine heart studied in this thesis were freshly excised, suspended in a plexiglass phantom filled with fluorinert (to eliminate artifacts) and placed in an MR head coil for ex-vivo imaging. All DW-MR studies were performed on a dedicated 1.5T GE Signa Excite scanner using a custom FSE pulse sequence. They used the following MR parameters: 
\begin{itemize}
    \item $TE = 35$ ms
    \item $TR = 700$ ms
    \item echo train length $=2$
    \item $b = 0$ for the unweighted MR images. As a reminder, the b factor is explained in \ref{dw_imaging}
    \item $b = 500 $ s/$\text{mm}^2$ for the images where the 7 diffusion gradients were applied
    \item $256 \times 256$ k-space
    \item Field of View $FOV = 10-16$ cm
    \item Slice Thickness $ = 1.2$ mm, yielding a sub-millimetric voxel size
\end{itemize}
From each heart, select samples containing an infarct were cut to align with the short-axis view of the MR images and prepared for histopathology to confirm the collagen deposition in the infarct area.

\begin{figure}[h!]
    \centering
    \includegraphics[width=\textwidth]{figures/Gip4_histology}
    \caption{Histology of a slice of infarcted heart nb 4}
    \label{fig:histology_pig_4}
\end{figure}

One of the difficulties that remains is to try and find a good correspondence between the slices from the histology and the ones that were obtained from the imaging that occured before the histopathology was performed.  \ref{fig:histology_pig_4} shows a histology image of a slide from an infarcted heart, with intact myocytes in the normal tissue and altered tissue microstructure in the infarcted zone. As depicted by the Masson Trichrome stain, the ischemic border zone (BZ) had collagen fibrils interdigitated between viable myocytes. In the dense scar area, necrotic myocytes were completely replaced by mature fibrosis (the final product of collagen degradation), resulting in the loss of myocardial anisotropy.

\section{Source of input data}

The data we have been using are DT-MRI of porcine hearts, both healthy for comparison and infarcted. We have had access to 10 hearts including 2 healthy. For every infarcted heart, the imaging process took place 6 weeks after infarct and histology was done after that for some of the cases - as it is an expensive process - which allows us to compare our observations to the ground truth. Here is an overview of the data we had access to and its usability:

\begin{center}
    \begin{tabular}{|c | c | c | c|} 
         \hline
         \shortstack{Heart \\ (Acquisition Date)} & Infarcted or Healthy & Region of infarct \\
         \hline
         2 & Infarcted & \shortstack{Left Circumflex artery \\ (LCX)} \\ 
         \hline
         4 & Infarcted & \shortstack{Left Anterior Descending \\ (LAD)} \\
         \hline
         5 & Infarcted & LCX \\
         \hline
         6 & Infarcted & LCX \\
         \hline
         7 & Infarcted & \shortstack{Rigth Coronary Artery \\ (RCA)} \\ 
         \hline
         17 & Infarcted & Bad quality \\
         \hline
         18 & Infarcted & Bad quality \\
         \hline
         23 & Infarcted & Unreadable \\
         \hline
         25 & Healthy & - \\
         \hline
         28 & Healthy & Too Noisy \\
         \hline
    \end{tabular}
\end{center}

\section{Quality of our data}

For a better and more accurate analysis of our results, we will focus mostly on the best quality data (least noisy) we had access to and fortunately those include one control heart (\#5: \ref{fig:pig25}) and 5 infarcted hearts (\#2: \ref{fig:pig2}, \#4: \ref{fig:pig4}, \#5: \ref{fig:pig5}, \#6: \ref{fig:pig6}, \#7: \ref{fig:pig7}).

Due to the imaging technology used, and to the fact that a fairly low Tesla intensity was used to acquire this data (1.5T), a good amount of noise is present in the raw data. We have taken advantage of the knowledge on the type of noise (Rician noise) that is present in MRI-acquired data to apply non-local denoising strategies which are a bit expensive in time but can smooth rather efficiently and not by too much the data to be able to have a better readability of the fiber orientation without oversmoothing. This prevents this strategy from smoothing even regions where the data is chaotic due to the infarct or proximity to the boundary and not only because of noise issues from acquisition. A fine tuning of the smoothing strategy allowed us to get good quality datasets without removing relevant information in the regions of interest around the infarcts.

\section{Usage of MedInria software to read the DICOMs}

We took the advantage of the MedInria tool to load the DICOM images, process the data through the Rician denoising method and finally use their Diffusion Tensor Imaging tool to compute the value of the tensor matrix \ref{diffusion_tensor_matrix} at each voxel location in the heart.

From the tensor matrices at every location, we were easily able to compute the fiber orientation at every location with a potential flip in the sense of the vector. Cylindrical consistency was used to enforce a consistent sense throughout the heart.

Once we have the fiber orientation at every voxel, we can use our previous work to compute the connection forms and get an approximative appreciation of the quality of the fitting, depending on hyper parameters that have to be fine tuned by mostly on the coherence of the underlying data.
\chapter{Methodology}

\section{Modeling fiber geometry using connection forms}

The histograms of $c_{1,2,3}$ for these cases are discussed in Section \ref{histogram_section}.

\begin{figure}[h!]
    \centering
    \begin{subfigure}[h!]{0.48\textwidth}
        \centering
        \includegraphics[width=\textwidth]{figures/pig4_c123_slice_19}
        \caption{$c_{1,2,3}$ (infarcted)}
        \label{fig:c123infarcted}
    \end{subfigure}
    \hfill
    \begin{subfigure}[h!]{0.48\textwidth}
        \centering
        \includegraphics[width=\textwidth]{figures/pig25_c123_slice_30}
        \caption{$c_{1,2,3}$ (healthy)}
        \label{fig:c123healthy}
    \end{subfigure}
    \caption{$c_{1,2,3}$ with range of values in porcine hearts}
    \label{fig:c123all}
\end{figure}

Connection forms measure the local rotations of the frame axes $\mathbf{f}_1, \mathbf{f}_2, \mathbf{f}_3$. Here we focus on the contraction of the 1-form $c_{1,2}$ on the frame axes $\mathbf{f}_3$ and compare its values in a short axis slice of a pig heart with an infarct \ref{fig:c123infarcted} with those in a short axis slice from a healthy pig heart \ref{fig:c123healthy}.\\
On a qualitative approach, looking at the colors and comparing the smoothness of figures \ref{fig:c123infarcted} with the control health \ref{fig:c123healthy}, we can already notice how in the healthy case and in regions remote from the infarct in the infarcted case we have smooth and regular colors which represent a rather constant value of $c_{1,2,3}$. Whereas in the infarct region, it is clear that the values vary a lot more and give a strong impression of a lack of coherence.\\
A more quantitative approach will be discussed in Section \ref{histogram_section}.
 
 \begin{figure}
     \centering
     \includegraphics[width=\textwidth]{figures/pig4_error_of_fit_slice_19}
     \caption{Error of fit, giving the absolute angle difference between our estimation from the connection forms and the ground truth}
     \label{fig:error_of_fit}
 \end{figure}
 
Cartan frame field analysis applies to smoothly rotating frame fields. In the presence of infarcts fiber orientation coherence is lost. The connections then fail to explain the orientation of fibers in a local neighborhood and fitting errors using this method go up \ref{fig:error_of_fit}. This association of frame field fitting error with fiber incoherence is the key insight behind the developments in this thesis.

\section{Cartan Frame Fitting and Error Analysis}

As explained earlier, at each voxel we use an estimate of the fiber orientation given by the orientation of the first principal eigen vector of a diffusion tensor reconstruction for $\mathbf{f}_1$. We then estimate the heart wall normal as the gradient of the distance function to the boundary of the myocardium and take the component of the normal that is orthogonal to $\mathbf{f}_1$ to be $\mathbf{f}_3$. $\mathbf{f}_2$ is then taken to be their cross product. Our numerical implementation of frame field fitting relies on finding the best estimates of the 9 connections at each voxel in the sense of explaining the orientations in its neighborhood. Specifically, using Nelder-Mead optimization, we minimize an energy given by the angle between the measured orientation at each neighbor and that given by rotating the frame field by a particular set of connections. Once this method converges the error of fit at a voxel is taken to be the average angular error between fiber orientations in a neighborhood and those given by rotating the frame at that voxel using its connection parameters.
\newcommand{\degree}{^{\circ}}

\chapter{Results and Discussion}

We used an established Rician smoothing method to deal with noise in the diffusion images \cite{wiest2008rician}. This is a type of non-local smoothing technique which uses voxel to voxel similarity to guide the smoothing process. In general the parameters for this filtering method must be tuned to prevent oversmoothing, which can happen for instance if the weight is not sufficiently penalized when the similarity is not high enough. We used the publicly available MedInria software to carry out the filtering, and then to reconstruct the diffusion tensor from the raw diffusion weighted scans, from which fiber orientations were extracted as the first principal eigen vector. In practice we used a threshold on the FA map as a mask to restrict further processing.

\section{Quantitative Results}

\begin{figure}
    \centering
    \begin{subfigure}{1\textwidth}
        \centering
        \includegraphics[width=\textwidth]{figures/histogram_pig6_no_smooth}
        \caption{Without any smoothing}
        \label{fig:histogram_pig6_no_smooth}
    \end{subfigure}
    \begin{subfigure}{1\textwidth}
        \centering
        \includegraphics[width=\textwidth]{figures/histogram_pig6_smooth}
        \caption{With Rician smoothing}
        \label{fig:histogram_pig6_smooth}
    \end{subfigure}
    \caption{Histograms of error of fit and connection form parameters for pig \# 6}
    \label{histo_pig6}
\end{figure}

Given the association between ADC and our error of frame fit, it is natural to compare these measures quantitatively throughout the myocardium. We did so for the 5 infarcted porcine hearts we analyzed by computing Dice coefficients to describe the overlap, in the following manner:

For the same heart let A be the set of voxels with ADC value > 0.6 and let B be the set of voxels with error of fit $> 15 \degree$. We computed the standard Dice coefficient $\frac{A \cap B}{A \cup B}$ as well as a modified coefficient $\frac{A \cap B}{A}$.
\begin{center}
    \begin{tabular}{|c | c | c |}
         \hline
         Heart & $\frac{A \cap B}{A \cup B}$ & $\frac{A \cap B}{A}$ \\
         \hline
         2 & 0.40 & 0.80 \\ 
         \hline
         4 & 0.43 & 0.89 \\
         \hline
         5 & 0.47 & 0.76 \\
         \hline
         6 & 0.46 & 0.87 \\
         \hline
         7 & 0.27 & 0.94 \\ 
         \hline
    \end{tabular}
\end{center}

These results demonstrate that typically over 80\% of the locations with increased diffusion also yield a high error of fit using our frame fitting method, due to the loss of geometric coherence of fiber orientations. However there are additional locations where fiber orientations are not smooth, typically at the linings of the heart wall, or near the edges of a collapsing and narrow right ventricle. These are picked up by our error of fit measure but not by the ADC measure, likely because there is no increase in collagen there.

\subsection{Histograms of connection forms} \label{histogram_section}

Here are examples of the histograms that we computed on the entire heart for every usable example that we had. The histogram \ref{fig:histogram_pig6_smooth} gives us information on the distribution of $c_{i,j,k}, \forall (i, j, k) \in [1, 3]^3$ and the distribution of the error of fit for each frame axis $\mathbf{f}_1, \mathbf{f}_2$ and $\mathbf{f}_3$.

In this thesis we will focus on the most commonly used parameter for comparison, the helix angle $c_{1,2,3}$. Our tool gives us the mean and the mode (value most represented) of this value through the whole heart. As we have a Gaussian distribution of these values the mode is relevant and helping as it will represent mostly fibers in the core of the heart wall and not take into account (as the mean value does) extreme values that can occur on the boundaries of the heart wall.

The papillary muscles, to name only one side effect, which are mostly aligned in a top-down direction along the long axis, can have non negligible effect on the $c_{1,2,3}$ value and they are difficult to remove from our model as it can be difficult to determine where they start exactly and which layer should then be kept and which should be considered as papillary muscle and therefore removed.

In the table we regroup the results we obtained from all relevant datasets and analyze the total helix angle in degrees that we obtain the following way (where 57.3 is the conversion from rad to $\degree$):
\begin{equation}
    \text{total helix angle} = c_{1,2,3}\cdot \text{\#voxels}\cdot 57.3
\end{equation}
The values that we get are close to the theoretical value of $120\degree$ reported in the literature \cite{piuzephd}.

\begin{center}
    \begin{tabular}{|c | c | c | c| c| c|} 
         \hline
         Heart & \shortstack{Thickness \\ (\# voxels)} & \shortstack{$c_{1,2,3}$ mean \\ rad/voxel} & \shortstack{$c_{1,2,3}$ mode \\ rad/voxel} & \shortstack{Total helix angle \\ (using mean)} & \shortstack{Helix angle \\ (using mode)}\\
         \hline
         2 & 40 & -0.066 & -0.025 & -151.2$\degree$ & -57.3$\degree$ \\ 
         \hline
         4 & 45 & -0.084 & -0.047 & -216.6$\degree$ & -121.2$\degree$ \\
         \hline
         5 & 30 & -0.048 & -0.020 & -82.5$\degree$ & -34.38$\degree$ \\
         \hline
         6 & 30 & -0.069 & -0.053 & -118.6$\degree$ & -91.1$\degree$ \\
         \hline
         7 & 45 & -0.077 & -0.053 & -198.5$\degree$ & -136.6$\degree$ \\ 
         \hline
         25 & 35 & -0.072 & -0.034 & -144.4$\degree$ & -68.2$\degree$ \\
         \hline
    \end{tabular}
\end{center}


\section{Qualitative Results}

\subsection{Importance of the filtering from the fibers...}

In topview figures \ref{fig:pig4_topviews} and a more transverse view where the helix angle is easier to visualize \ref{fig:pig4_helix}, we have a first glimpse at the difference of the fiber orientation and cohesion before and after smoothing the raw data through a Rician filter, which is the appropriate filtering method when dealing with DMRI data.

The region shown in figures \ref{fig:pig4_helix_no_smooth} and \ref{fig:pig4_helix_smooth} is a region away from the infarct where we should see a smooth turning of fibers. This region corresponds to the top-right region of the same slice presented in figures \ref{fig:pig4_topview_no_smooth} and \ref{fig:pig4_topview_smooth}.

Tracing fibers from raw data leads to an extremely noisy data from which it is hard to infer any general structure, which we will see in \ref{fig:histogram_pig6_no_smooth}. Indeed the low magnetic field value (1.5T) is one of the reasons why the data is not of the highest quality, as well as the low number of iterations in the acquisition process.

Thanks to the Rician smoothing it is again possible to guess the helix angle turning from outer wall to inner wall although this is still not perfect, as we want to keep as much information as possible with using the lightest filtering possible.

\begin{figure}[h!]
    \centering
    \begin{subfigure}{.48\textwidth}
        \includegraphics[width=\textwidth]{figures/pig4_topview_no_smooth}
        \caption{Without any smoothing}
        \label{fig:pig4_topview_no_smooth}
    \end{subfigure}
    \begin{subfigure}{.48\textwidth}
        \includegraphics[width=\textwidth]{figures/pig4_topview_smooth}
        \caption{With Rician smoothing}
        \label{fig:pig4_topview_smooth}
    \end{subfigure}
    \caption{Top view of heart fibers for pig 4}
    \label{fig:pig4_topviews}
    \begin{subfigure}{.48\textwidth}
        \includegraphics[width=\textwidth]{figures/pig4_helix_no_smooth}
        \caption{Without any smoothing}
        \label{fig:pig4_helix_no_smooth}
    \end{subfigure}
    \begin{subfigure}{.48\textwidth}
        \includegraphics[width=\textwidth]{figures/pig4_helix_smooth}
        \caption{With Rician smoothing}
        \label{fig:pig4_helix_smooth}
    \end{subfigure}
    \caption{View of the helix angle of heart fibers for pig 4}
    \label{fig:pig4_helix}
\end{figure}

\subsection{...To the tractography and fitting process}

Using the fiber orientation that we can get from the computed tensor matrix, we can run tractography on our results. This process gives us an idea of how fibers should be wrapped around the heart and work together in the muscle structure. Looking at the tractography run on the raw data \ref{fig:pig4_tracto_no_smooth} it is hard to see where the infarct would be by just looking at the tracing of existing fibers and trying to infer their direction. The infarct region, as explained in \cite{wu2007mr}, should be the region with the least coherence in its fiber directions whereas regions remote from the infarct should not be affected in their structure.

\begin{figure}[h!]
    \centering
    \begin{subfigure}{.48\textwidth}
        \includegraphics[width=\textwidth]{figures/pig4_topview_tracto_no_smooth}
        \caption{Without any smoothing}
        \label{fig:pig4_tracto_no_smooth}
    \end{subfigure}
    \begin{subfigure}{.48\textwidth}
        \includegraphics[width=\textwidth]{figures/pig4_topview_tracto_smooth}
        \caption{With Rician smoothing}
        \label{fig:pig4_tracto_smooth}
    \end{subfigure}
    \caption{Tractography of heart fibers for pig 4}
    \label{fig:pig4_tractos}
\end{figure}

Looking at the tractography run on the Rician smoothed data \ref{fig:pig4_tracto_smooth} on the other hand gives a clear hint on the region where the infarct is present (LAD) whereas remote regions do not seem affected by this and wrap smoothly around the heart.

Finally, looking at histograms \ref{fig:histogram_pig6_no_smooth} and \ref{fig:histogram_pig6_smooth} we can get from our procedure to look at statistics on the visualized information presented, we see several clear indications that smoothing will be a capital step in our further analysis of our datasets:
\begin{enumerate}
    \item Shape of Gaussians we get from the datasets. The center is much better defined in for the smoothed data \ref{fig:histogram_pig6_smooth} as well as overall shape which is less spiky
    \item The number of voxels that count in our computations: from $1564$ with raw unfiltered data \ref{fig:histogram_pig6_no_smooth} to $59187$ after the Rician filter \ref{fig:histogram_pig6_smooth}
\end{enumerate}
The number of voxels that count in our computations is the number of voxels where we could converge to a value of connection form matrix $(c_{i, j, k})_{(i, j, k) \in \mathbb{R}^3}$. Voxels where we could not converge to a value after a given number of iterations were discarded in the histogram to limit the noise importance in the final observations.

\subsection{Error of fit vs ADC and FA}

\begin{figure}[h!]
    \centering
    \begin{subfigure}{.31\textwidth}
        \includegraphics[width=\textwidth]{figures/pig2_adc_31}
        \caption{ADC map}
        \label{fig:pig2_adc}
    \end{subfigure}
    \begin{subfigure}{.31\textwidth}
        \includegraphics[width=\textwidth]{figures/pig2_err_31}
        \caption{Error of fit}
        \label{fig:pig2_err}
    \end{subfigure}
    \begin{subfigure}{.31\textwidth}
        \includegraphics[width=\textwidth]{figures/pig2_fa_31}
        \caption{FA map}
        \label{fig:pig2_fa}
    \end{subfigure}
    \caption{Pig 2: Error of fit computed using our framework compared to ADC and FA map}
    \label{fig:pig2}


    \begin{subfigure}{.31\textwidth}
        \includegraphics[width=\textwidth]{figures/pig4_adc_21}
        \caption{ADC map}
        \label{fig:pig4_adc}
    \end{subfigure}
    \begin{subfigure}{.31\textwidth}
        \includegraphics[width=\textwidth]{figures/pig4_err_21}
        \caption{Error of fit}
        \label{fig:pig4_err}
    \end{subfigure}
    \begin{subfigure}{.31\textwidth}
        \includegraphics[width=\textwidth]{figures/pig4_fa_21}
        \caption{FA map}
        \label{fig:pig4_fa}
    \end{subfigure}
    \caption{Pig 4}
    \label{fig:pig4}
    
    \begin{subfigure}{.31\textwidth}
        \includegraphics[width=\textwidth]{figures/pig5_adc_22}
        \caption{ADC map}
        \label{fig:pig5_adc}
    \end{subfigure}
    \begin{subfigure}{.31\textwidth}
        \includegraphics[width=\textwidth]{figures/pig5_err_22}
        \caption{Error of fit}
        \label{fig:pig5_err}
    \end{subfigure}
    \begin{subfigure}{.31\textwidth}
        \includegraphics[width=\textwidth]{figures/pig5_fa_22}
        \caption{FA map}
        \label{fig:pig5_fa}
    \end{subfigure}
    \caption{Pig 5}
    \label{fig:pig5}
\end{figure}
    
\begin{figure}[h!]
    \centering
    \begin{subfigure}{.31\textwidth}
        \includegraphics[width=\textwidth]{figures/pig6_adc_24}
        \caption{ADC map}
        \label{fig:pig6_adc}
    \end{subfigure}
    \begin{subfigure}{.31\textwidth}
        \includegraphics[width=\textwidth]{figures/pig6_err_24}
        \caption{Error of fit}
        \label{fig:pig6_err}
    \end{subfigure}
    \begin{subfigure}{.31\textwidth}
        \includegraphics[width=\textwidth]{figures/pig6_fa_24}
        \caption{FA map}
        \label{fig:pig6_fa}
    \end{subfigure}
    \caption{Pig 6}
    \label{fig:pig6}
    
    \begin{subfigure}{.31\textwidth}
        \includegraphics[width=\textwidth]{figures/pig7_adc_14}
        \caption{ADC map}
        \label{fig:pig7_adc}
    \end{subfigure}
    \begin{subfigure}{.31\textwidth}
        \includegraphics[width=\textwidth]{figures/pig7_err_14}
        \caption{Error of fit}
        \label{fig:pig7_err}
    \end{subfigure}
    \begin{subfigure}{.31\textwidth}
        \includegraphics[width=\textwidth]{figures/pig7_fa_14}
        \caption{FA map}
        \label{fig:pig7_fa}
    \end{subfigure}
    \caption{Pig 7}
    \label{fig:pig7}
    
    \begin{subfigure}{.31\textwidth}
        \includegraphics[width=\textwidth]{figures/pig25_adc_30}
        \caption{ADC map}
        \label{fig:pig25_adc}
    \end{subfigure}
    \begin{subfigure}{.31\textwidth}
        \includegraphics[width=\textwidth]{figures/pig25_err_30}
        \caption{Error of fit}
        \label{fig:pig25_err}
    \end{subfigure}
    \begin{subfigure}{.31\textwidth}
        \includegraphics[width=\textwidth]{figures/pig25_fa_30}
        \caption{FA map}
        \label{fig:pig25_fa}
    \end{subfigure}
    \caption{Pig 25}
    \label{fig:pig25}
\end{figure}

These figures give us a good understanding and overview of how the error of fit can provide supplementary information compared to what the ADC and FA map can give us.

If we focus on the first example \ref{fig:pig2}, we can clearly see in the ADC map \ref{fig:pig2_adc} the region of the infarct where the value is the highest (yellow region); and it really gives the impression of a scar region that contains almost only collagen and no fibrous tissue whatsoever given the apparent anisotropy. But if we look at the FA map \ref{fig:pig2_fa} we can start wondering if there is not a region close to the endocardium where fibers have either resisted or been reconstructed after the infarct in this region.

As for the error of fit, it gives a clear indication of the existence of layers of fibers near the endocardium that still makes the connection with the rest of the heart possible and the contraction, even if badly hindered by the infarct and loss of muscle fibers, still possible.

This arrangement of the surviving fibers as described in the literature overviewed in \ref{surviving_fibers} is also clearly visible in the examples displayed in \ref{fig:pig4} and \ref{fig:pig5}.

The example \ref{fig:pig6} is also a very interesting sample to compare our method to, and it shows that in some cases as mentioned in \cite{ursell1985structural}, intramural muscles seem to be completely absent in this case and can gravely hinder the contraction of the heart.

Our approach seems to be more tolerant in the exhibition of surviving fibers in the infarct region and in some cases show how CR can make the infarct region still a little contractile.

The last provided example \ref{fig:pig25} is used as a control healthy heart to make sure of the extent to which we can use our results. We can clearly see in this healthy example that the error of fit is low pretty much everywhere, except in regions near the heart wall (endothilial and epithilial) and the region where the Left Ventricule (LV) and the Right Ventricule (RV) connect to each other. This is due to difficulty in our method to compute connection forms in areas where fibers tend to go in different directions, depending if they work towards the LV contraction, the RV contraction or else. Then the method performs less well to try to fit smoothly varying geometrical fiber structures.

\subsection{Discussion}

In all the examples seen in the previous section, locations where the ADC value is high are typically also ones where the error of fit is high, with regions of low error of fit being restricted to healthy tissue. In addition though, the error of fit is also consistently high at locations close to the epithelial and endothelial boundaries, signaling a loss in geometric coherence of fibers there. Curiously this phenomena is also seen at the borders of the healthy heart \ref{fig:pig25_err}.

The error of fit shows information about the existence of surviving fibers in the infarct and border zone that can help the contractile function of the heart after the infarct. This is precious knowledge as it is a non-invasive way to qualify the capacity of the heart to keep beating and pumping blood to the body although the heart has suffered from an infarct. Further experiments could be made to support this analysis and show how the efficiency of an infarcted heart is related to the remaining fibers by conducting blood flow experiments before the imaging process.

Another very promising research topic would be to analyze the evolution of surviving fibers at different time frames after infarct, as we were limited to the datasets available which were all 6 weeks after infarct.



\bibHeading{References}
\bibliographystyle{plain}
\bibliography{myrefs}


\end{document}







