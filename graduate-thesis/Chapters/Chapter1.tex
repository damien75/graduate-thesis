\chapter{Introduction}

The technological improvement of noninvasive medical imaging techniques in the past couple of decades has allowed us to get a better understanding of the heart's structure \cite{sinusas2008multimodality}. This is also thanks to the increasing amount of data that can be collected more and more easily and with better definition as time goes by. \\
As the imaging techniques get better \cite{shaw2010cardiovascular}, the analysis of the images we get still plays a major role in understanding the causes and effects of the disease and has to be used in a clever way to drive further down the number of cardiovascular diseases that is still the leading global cause of death \cite{mozaffarian2015heart} since 1900 (except for the year 1918). \\
Although the variety of imaging modalities is very wide and information can be convened from these multiple sources, the most valuable for a clinician will be a fast processed and easy to understand and visualize data that we hope our imaging tools of fibers can provide. The visualization of those in 3D is key since they are the pillar of the mechanical behavior of the heart \cite{hooks2002cardiac}.\\
The core organization of these fibers can be altered by various common pathologies and lead to a deficient contraction of the heart \cite{beg2004computational}. \\
Numerous former histological studies have shown how fibers in a healthy heart are smoothly wrapped around it and that the variation of the helix angle (measure of fiber orientation) is a common descriptor of local fiber geometry \cite{geerts2002characterization}. Computational aided visualization of the disorganization that can occur in the heart structure could be helpful for modern treatments that feature tissue engineering methods to restore the contractile properties of the heart \cite{caplan2006mesenchymal,laflamme2007cardiomyocytes,laflamme2005regenerating, song2012heart, zimmermann2004engineered}, or in more extreme cases where they proceed with reconstruction, or even ablation of infarcted regions \cite{athanasuleas2004surgical, di2001effects, jones2009coronary, sartipy2005dor}.

\section{Objectives of this Thesis}

In this thesis we start from our knowledge of Maurer-Cartan connection form description of the fibers of the heart \cite{pami2015}. From this basis we try and analyze the possible impacts that an infarct can have on the fiber orientation in the infarct area and areas away from the infarct regions.