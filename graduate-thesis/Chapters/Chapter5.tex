\chapter{Results and Discussion}

We used an established Rician smoothing method to combat noise in the diffusion images \cite{wiest2008rician}. This is a type of non-local smoothing technique which uses voxel to voxel similarity to guide the smoothing process. In general the parameters for this filtering method must be tuned to prevent oversmoothing, which can happen for instance if the weight is not sufficiently penalized when the similarity is not high enough. We used the publicly available MedInria software to carry out the filtering, and to then reconstruct the diffusion tensor from the raw diffusion weighted scans, from which fiber orientations were extracted as the first principal eigen vector. In practice we used a threshold on the FA map as a mask to restrict further processing.

\section{Qualitative Results}

Supplementing the earlier results in Fig. 1, Fig 3 compares ADC maps (top row) with our Cartan frame fitting-based error of fit maps in degrees (second row) for a healthy pig heart (left column) and 2 additional infarcted hearts (middle and right columns). In all these examples locations where the ADC value is high are typically also ones where the error of fit is high, with regions of low error of fit being restricted to healthy tissue. In addition though, the error of fit is also consistently high at locations close to the epithelial and endothelial boundaries, signaling a loss in geometric coherence of fibers there. Curiously this phenomena is also seen at the borders of the healthy heart (Fig. 3 left column).

\section{Quantitative Results}

Given the association between ADC and our error of frame fit, it is natural to compare these measures quantitatively throughout the myocardium. We did so for the 5 infarcted porcine hearts we analyzed by computing Dice coefficients to describe the overlap, in the following manner. For the same heart let A be the set of voxels with ADC value > 0.6 and let B be the set of voxels with error of fit > 15 degrees. We computed the standard Dice coefficient A∩B/A∪B as well as a modified coefficient A∩B/A. These results, shown in Table. 1 (left), demonstrate that typically over 80\% of the locations with increased diffusion also yield a high error of fit using our frame fitting method, due to the loss of geometric coherence of fiber orientations. However there are additional locations where fiber orientations are not smooth, typically at the linings of the heart wall, or near the edges of a collapsing and narrow right ventricle. These